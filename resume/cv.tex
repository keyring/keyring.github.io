%!TEX TS-program = xelatex
\documentclass[full]{rvca}

% Compile with XeLatex

\begin{document}

% % % % % % % % % % % Begin Side Panel


\begin{aside} % each new line forces a line break
\header{\AvatarIcon}{曾幸福}{游戏客户端开发}
\section{\uppercase{contacts}}
+86 13550338094 {\faMobile}
keyrings@163.com {\scriptsize\faLink}
zkeyring@gmail.com {\scriptsize\faLink}
\href{http://www.photoneray.com}{\color{sidebandtextcolor}www.photoneray.com \faGlobe} 
\href{https://github.com/keyring}{\color{sidebandtextcolor}@keyring \faGithub}
\ghost
% \QrCode
\section{\uppercase{interests}}
游戏设计\\游戏引擎\\图形渲染\\底层优化
%
\ghost
\section{\uppercase{skills}}
{\scriptsize\faHeart}Cocos2d-X, Unreal, UI, Git, Emacs, OpenGL, OpenCL, Research, Writing.
\ghost 
\section{\uppercase{human code}}
英语,普通话,四川话
\ghost
\section{\uppercase{computer code}}
{\scriptsize\faHeart}C, C++, {\scriptsize\faHeart}Lua, Python, Shell, Swift, Java, Markdown, Scheme, Emacs Lisp, \LaTeX
\ghost
\end{aside}


% % % % % % % % % % Begin Main Content

\begin{statement}
毕业于云南大学软件学院数字媒体技术专业,后加入创业公司完整参与一款游戏的相关开发工作。在工具开发、游戏客户端开发、图形学原理、游戏引擎、iOS开发等领域均有涉猎,擅长高效学习与解决问题。当前主要探索于\textbf{实时渲染}和\textbf{游戏设计}。
\end{statement}

\subsection{}

\section{Key Achievements}

\achievements
{公司项目:从脚本工具的编写到编辑器的扩展;从完善现有代码到独立设计并实现多个系统;从指导美术配置输出流程到自动化资源压缩打包;从提升团队效率到优化游戏效率,从项目框架搭建到最终上线,该项目所涉及到的所有领域均有参与并发挥作用。}
{\href{http://a.app.qq.com/o/simple.jsp?pkgname=com.kode.Thirteen}{独立游戏}:独立创作的小游戏,负责游戏设计,技术开发,艺术创作,测试反馈,对接各大市场并最终上架销售。项目采用 Git 版本控制,利用分支分渠道集成功能,如免费版集成 Admob 广告,iOS 版集成 Game Center,国内 Android 版集成 QQ 分享等等。如今游戏已上线并有后续开发计划。}
{\href{https://github.com/keyring/point}{Point Software Renderer}:技术学习项目。因疑惑现代图形学API的封装,欲掌握计算机图形学的本质,于是使用 C 语言做图形学基本算法的实现,并采用ppm文件展示绘制结果。}
{\href{http://github.com/keyring}{GitHub\&开源}:个人项目均托管于 GitHub,比如平时学习与开发中的所思所想便记于\href{http://www.photoneray.com}{博客}上。闲时也向其他\href{https://github.com/cloudwu/lua53doc/graphs/contributors}{开源项目}做贡献。}
{}

\section{Work}

% \workitem
% {}%Company
% {}%Years 
% {}%Title 
% {}%Dates
% {}%Responsibilities
% {}%Achievement(1)
% {}%Achievement(2) %optional
% {}%Achievement(3) %optional

\workitem{飞鱼成都 光橙互动}%Company
{2014--present}%Dates
{游戏客户端开发工程师} %Title 
{利用Cocos2d-X游戏引擎做游戏客户端开发。负责部分游戏UI开发,游戏功能开发(新手引导、副本、剧情等),部分关键系统的设计与实现,内部工具链的维护扩展以及第三方SDK接入和游戏优化。}%Responsibilities
% Achievement
{设计与实现 spine 换装系统。利用 spine 骨骼数据与贴图分离的特点,动态替换局部贴图并修改 spine runtime 最终达到人物换装的效果。同时规范美术出图流程,编写皮肤分类打包工具,实现自动化换装资源添加。}
{工具链的维护与扩展。优化与扩展地图编辑器,新增预览功能;搭建剧情编辑框架并指导设计人员独立编写剧情脚本;灵活运用shell,python,Lua编写脚本工具方便美术与策划的工作,如技能数值调整,图片压缩打包,换装实时预览等。}
{SDK接入。灵活运用工具和代码将多种SDK顺利部署于多端(iOS,Android),涉及到多种语言(C++,Lua,OC,Java)的转换衔接。}
{游戏优化。合理利用工具(instrument)分析游戏,对内存,渲染,资源配置等方面进行优化,使得游戏在最低配置的测试机上运行帧数稳定50+。}
{}

\section{Education}

\datedsubsection{云南大学}{2010--2014}
数字媒体技术专业,学士学位


\end{document} ​
