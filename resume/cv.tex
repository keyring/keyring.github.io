%!TEX TS-program = xelatex
\documentclass[full]{rvca}

% Compile with XeLatex

\begin{document}

% % % % % % % % % % % Begin Side Panel


\begin{aside} % each new line forces a line break
\header{\AvatarIcon}{曾幸福}{游戏客户端工程师}
\section{\uppercase{contacts}}
+86 13550338094 {\faMobile}
keyrings@163.com {\scriptsize\faLink}
zkeyring@gmail.com {\scriptsize\faLink}
\href{http://www.photoneray.com}{\color{sidebandtextcolor}www.photoneray.com \faGlobe} 
\href{https://github.com/keyring}{\color{sidebandtextcolor}@keyring \faGithub}
\ghost
\section{\uppercase{interests}}
Game Dev\\Game Engine\\Graphics\\AR/VR
\ghost
\section{\uppercase{skills}}
{\scriptsize\faHeart}Cocos2d-X, Unreal, Git, Emacs, OpenGL, Writing
\ghost 
\section{\uppercase{human code}}
English, Chinese
\ghost
\section{\uppercase{computer code}}
{\scriptsize\faHeart}Lua, {\scriptsize\faHeart}C, C++, {\scriptsize\faHeart}Markdown, Python, \LaTeX
\ghost
\section{\uppercase{creative}}
\href{http://a.app.qq.com/o/simple.jsp?pkgname=com.kode.Thirteen}{\color{sidebandtextcolor}Thirteen}
\href{https://github.com/keyring/point}{\color{sidebandtextcolor}Point}
\href{http://www.photoneray.com/Tear}{\color{sidebandtextcolor}Tear Engine}
\href{https://github.com/keyring/ShaderToy}{\color{sidebandtextcolor}ShaderToy}
\ghost
\end{aside}


% % % % % % % % % % Begin Main Content

\begin{statement}
毕业于云南大学数字媒体技术专业,至今供职过两家游戏公司,任游戏开发核心岗位,擅长高效学习与解决问题。当前主要负责\textbf{游戏开发}、\textbf{架构设计}与\textbf{渲染优化}。
\end{statement}

\subsection{}

\section{Key Achievements}

\achievements
{2D动作卡牌游戏。使用\textbf{Cocos2d-X}引擎完整参与整个项目开发,涉及架构设计、核心系统开发、美术规范制定、自动化脚本、渲染优化以及后期SDK接入。}
{德州扑克棋牌游戏。使用公司自研引擎做项目迭代与优化,负责功能迭代、版本同步、代码重构优化以及脚本工具。}
{3D横板射击游戏。使用\textbf{Unreal Engine 4}独立制作的3D横板射击游戏,包括世界架构、系统设计、关卡设计以及C++、Blueprint、Persona、Cascade等UE4功能的配合使用。}
{}

\section{Work}

% \workitem
% {}%Company
% {}%Years 
% {}%Title 
% {}%Dates
% {}%Responsibilities
% {}%Achievement(1)
% {}%Achievement(2) %optional
% {}%Achievement(3) %optional
\workitem{深圳市东方博雅科技有限公司}
{2016.3--至今}
{引擎客户端工程师}
{采用公司自研引擎维护项目,涉及功能迭代、代码重构、效率优化、工作流优化、技术培训等等}
{代码重构:制定代码书写规范,逐步统一风格;梳理重要业务逻辑并重新实现,力求简短清晰;整理代码目录结构,职责明确,突出模块化}
{效率优化:优化引擎某些效率问题突出的控件,如列表、高斯模糊等;针对批渲染特性,调整代码结构与布局结构,降低DC;使用工具profile性能热点,逐个击破。}
{技术培训:普及游戏引擎与渲染的基础知识;推广新的规范、框架以及新引擎;分享Lua语言特性和技巧;示范新工具的使用如Git,markdown;书写各类文档,知识沉淀。}
{}

\workitem{飞鱼科技·成都光橙互动}%Company
{2014.7--2016.3}%Dates
{游戏客户端工程师} %Title 
{利用Cocos2d-X游戏引擎做游戏客户端开发。负责游戏UI开发,游戏功能开发(新手引导、副本、剧情、换装等),工具维护扩展,SDK接入和游戏优化。}%Responsibilities
% Achievement
{换装系统。利用 spine 骨骼与贴图分离的特点,动态替换局部贴图并修改 spine runtime 实现人物换装;规范美术工作流,编写皮肤分类打包工具,实现完整换装流程。}
{工具链开发。优化地图编辑器,新增预览功能;搭建剧情编辑框架并指导设计师独立编写剧情脚本;编写辅助工具提高工作效率,如图片压缩、换装预览、剧情预览等。}
{游戏优化。利用工具(Instrument/Adreno GPU Profiler)分析游戏,定位性能热点,在内存,渲染,资源配置等方面进行优化。}
{}

\section{Education}

\educationitem{云南大学}{2010.9--2014.7}
{\emph{软件学院}}
{数字媒体技术专业} {学士学位}


\end{document} ​